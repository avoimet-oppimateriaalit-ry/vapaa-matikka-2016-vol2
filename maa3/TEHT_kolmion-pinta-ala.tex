\begin{tehtavasivu}

\paragraph*{Opi perusteet}

\begin{tehtava}
Laske kolmion ala, kun sen kanta ja korkeus ovat
\alakohdat{
	§ $ 7,6$
	§ $ \frac{6}{5}, \frac{4}{3}$
	§ $ \sqrt{13}, \sqrt{13}$,
}
vastaavasti.
\begin{vastaus}
\alakohdat{
	§ $21$
	§ $\frac{4}{5} = 0,8$
	§ $\frac{13}{2} = 6,5$
}
\end{vastaus}
\end{tehtava}

\begin{tehtava}
Kolmion $ABC$ ala on kaksi kertaa niin suuri kuin kolmion $DEF$, mutta sen korkeus on vain kolmasosa kolmion $DEF$ korkeudesta. Kuinka moninkertainen on sen on $ABC$:n kannan pituus verrattuna $DEF$ kantaan?
\begin{vastaus}
$ABC$:n kanta on kuusi kertaa niin pitkä kuin $DEF$:n kanta.
\end{vastaus}
\end{tehtava}


\begin{tehtava}
Laske kolmion ala, kun sen kaksi sivua, ja niiden välinen kulma on
\alakohdat{
	$ 3,4, 90^{\circ} $
	$ \sqrt{5}, 27, 60^{\circ} $
	$ \pi , 3, 45^{\circ}. $
}
\begin{vastaus}
\alakohdat{
	$6$
	$\frac{27\sqrt{15}}{4} \approx 26,1$
	$\frac{3\pi}{2\sqrt{2}} \approx 3,33$
}
\end{vastaus}
\end{tehtava}

\begin{tehtava}
Laske kolmion ala, kun sen sivujen pituudet ovat
\alakohdat{
	§ 3,4,5
	§ 2,3,4
	§ 1,2,3
}
\begin{vastaus}
\alakohdat{
	$6$
	$\frac{3\sqrt{15}}{4} \approx 2,90$
	Kolmion sivujen pituudet eivät voi olla 1, 2 ja 3 (kolmioepäyhtälö).
}
\end{vastaus}
\end{tehtava}

\paragraph*{Hallitse kokonaisuus}

\begin{tehtava}
Jos kolmion $ABC$ kaikki sivut ovat pidempiä kuin kolmion $DEF$ sivut, onko kolmion $ABC$ ala välttämättä suurempi kuin $DEF$:n ala?
\begin{vastaus}
Ei. Jos esimerkiksi kolmion $ABC$ sivujen pituudet ovat kukin 8 (eli kolmion on tasasivuinen), ja kolmion $DEF$ sivujen pituudet ovat 9, 9 ja 17, kolmion $ABC$ ala on suurempi kuin kolmion $DEF$ ala.
\end{vastaus}
\end{tehtava}

\begin{tehtava}
Muuten edellinen tehtävä, mutta tiedetään lisäksi, että kolmioilla on yksi yhteinen kulma.
\begin{vastaus}
Kyllä. Jos $\alpha$ on tämä yhteinen kulma, joka on kolmion $ABC$ ja $DEF$ sivujen ja $a$ ja $b$, ja $d$ ja $e$ välissä, vastaavasti. Nyt kolmion $ABC$ ala $= ab \sin \alpha > de \sin \alpha =$ kolmion $DEF$ ala. 
\end{vastaus}
\end{tehtava}

\begin{tehtava}
Todista, että 
\alakohdat{
§ suunnikkaan pinta-ala voidaan laske kaavalla \[A = ah \]
§ kolmion pinta-ala voidaan laskea kaavalla \[A = \frac{ah}{2} \]
}
\begin{vastaus}
\alakohdat{
§ [KUVA, jossa suunnikas jaettu korkeusjanoilla kolmeen osaan ja varjostettu siirrettävät alueet, you know, sellanen perus perustelu]
§ Käytä a)-kohtaa. Mitä saat, kun jaat suunnikkaan kahteen osaan? Entä käänteisesti?
}
\end{vastaus}
\end{tehtava}

\paragraph*{Sekalaisia tehtäviä}



\begin{tehtava}
Piste $P$ on kolmion $ABC$ sisällä. Suora $AP$ leikkaa sivun $BC$ pisteessä $D$. Todista, että
\[
\frac{BD}{CD} = \frac{|BDA|}{|CDA|} = \frac{|BDP|}{|CDP|} = \frac{|BPA|}{|CPA|}
\]
missä $|BDA|$ on kolmion $BDA$ ala jne.. Todista edelleen \termi{Cevan lause}{Cevan lause}: Jos pisteet $D$, $E$ ja $F$ ovat kolmion sivuilla $BC$, $CA$ ja $AB$ vastaavasti, suorat $AD$, $BE$ ja $CF$ kulkevat saman pisteen kautta jos ja vain jos
\[
\frac{BD\cdot CE \cdot AE}{CD \cdot EA \cdot FB}
\]
\begin{vastaus}
Vinkki: Kolmiolla $BDA$ ja $CDA$ on sama korkeus, vastaavasti kolmioilla $BDP$ ja $CDP$. Tutki verrantoja. $\Rightarrow$ Cevan lauseen suhteet voidaan esittää kolmoiden $BPA$, $APC$ ja $CPB$ alojen avulla. $\Leftarrow$ Tutki pistettä, $F'$ (janalla $AB$), jolla suorat kulkevat saman pisteen kautta ja käytä jo todistettua puolta todistaaksesi, että $F' = F$.
\end{vastaus}
\end{tehtava}

\end{tehtavasivu}
