\begin{tehtavasivu}

\paragraph*{Opi perusteet}

\begin{tehtava}
Todista, että kaikki tasasivuiset kolmiot ovat yhdenmuotoisia.
\end{tehtava}

\begin{tehtava}
Matti haluaa mitata jättiläisen nimeltä Pitkä Kissa, mutta huomaa, ettei mittanauha riitä. On lähes keskipäivä ja Matti keksii tavan mitata jättiläisen: Hän mittaa oman varjonsa olevan 42 senttimetriä pitkä ja jättiläisen varjon pituuden olevan 2,51 metriä. Matti tietää olevansa 1,80 metriä pitkä. Kuinka pitkä jättiläinen Pitkä Kissa on?

\begin{vastaus}
11 metriä
\end{vastaus}
\end{tehtava}

\begin{tehtava}
Kolmiot $ABC$ ja $ADE$ ovat yhdenmuotoiset, siten että pistekolmikot $A$, $B$ ja $D$, sekä $A$, $C$ ja $E$ ovat samalla suoralla, vastaavasti, ja $D$ on $A$:n ja $B$:n välissä.
\alakohdat{
§ Todista, että suorat $BC$ ja $DE$ ovat yhdensuuntaiset.
§ Todista, että
\[
\frac{AB}{AC} = \frac{AD}{AE} = \frac{BD}{CE}.
\]
} 
\end{tehtava}

\paragraph*{Hallitse kokonaisuus}
\begin{tehtava}
Puolisuunnikkaan $ABCD$ yhdensuuntaisten sivujen $AB$ ja $CD$ pituudet ovat $a$ ja $b$ $(a > b)$, vastaavasti. Pisteet $E$ ja $F$ ovat janoilla $AD$ ja $BC$, vastaavasti, niin että $AB$ ja $EF$ ovat yhdensuuntaisia. Janan $EF$ pituus on $k$.

\alakohdat{
§ Todista, että $E$ ja $F$ jakavat sivut $AD$ ja $BC$ samassa suhteessa. Mikä tämä suhde on ($k$:n funktiona)?
§ Jos $E$ on janan $AD$ keskipiste, määritä $k$
§ Olkoon lävistäjien $AC$ ja $BD$ leikkauspiste $P$. Jos $EF$ kulkee $P$:n kautta, missä suhteessa $E$ jakaa sivun $AD$? Mikä on tällöin $k$?
§ Jos $EF$ jakaa puolisuunnikkaan $ABCD$ kahteen yhdenmuotoiseen puolisuunnikkaaseen, määritä $k$.
%§ Mikä on $k$, jos kahden uuden puolisuunnikaan pinta-alat on sama?
}

\begin{vastaus}\alakohdat{
§ Vinkki: Jos $AD$:n ja $BC$:n leikkauspiste on $Q$, todista, että kolmiot $QAB$, $QEF$ ja $QDC$ ovat yhdenmuotoisia ja tutki verrantoja.
§ $k = \frac{a+b}{2}$
§ $\frac{AE}{ED} = \frac{a}{b}$, $k = \frac{2ab}{a+b}$
§ $k = \sqrt{ab}$
}
\end{vastaus}


\end{tehtava}

\begin{tehtava}
\alakohdat{
§ Kolmion $ABC$ sivujen pituudet ovat $AB = 8$, $AC = 6$ ja $BC = 7$. Sivulla $BC$
on piste $D$, jolle kulmat $BAD$ ja $DAC$ ovat yhtä suuret. Kuinka pitkät ovat janat $BD$
ja $DC$?
}
\begin{vastaus}
\alakohdat{
§ $BD = 4$, $DC = 3$
}
\end{vastaus}
\end{tehtava}

\paragraph*{Sekalaisia tehtäviä}

TÄHÄN TEHTÄVIÄ SIJOITTAMISTA ODOTTAMAAN

\end{tehtavasivu}
