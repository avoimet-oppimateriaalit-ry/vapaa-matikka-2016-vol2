\section*{Yhdenmuotoisuus}

Jos jokin tasokuvio on saatu toisesta tasokuviosta siirroilla, kierroilla tai
peilauksilla, niin näitä kuvioita sanotaan yhteneviksi. Koska pituudet ja kulmat säilyvät
näissä operaatioissa, näillä kuvioilla toisiaan vastaavat kulmat ja janat
ovat yhtä suuret.

Jos yhtenevää kuviota suurennetaan tai pienennetään, niin kulmat pysyvät edelleen yhtä suurina.
Vaikka pituudet eivät pysykään samoina, niin toisiaan vastaavien janojen pituuksien suhde
on sama kaikille janoille. Sanomme, että tässä tilanteessa kuviot ovat yhdenmuotoiset.

\laatikko{
	Kaksi kuviota ovat keskenään yhdenmuotoiset, jos niiden pisteille löytyy vastaavuus, jolle
	
	1) Kaikki vastinkulmat ovat keskenään yhtä suuret.
	
	2) Kaikkien vastinjanojen pituuksien suhde on vakio.
}

Aina yhdenmuotoisuuden tarkistaminen ei onnistu helposti. Onneksi kolmioille on yksinkertaisia
sääntöjä, joilla yhdenmuotoisuuden perusteleminen on erityisen helppoa.

\laatikko{
	\termi{kolmioiden yhdenmuotoisuussäännöt}{Kolmioiden yhdenmuotoisuussäännöt}
	
	Jos seuraavista ehdoista jokin on voimassa kolmioparille, niin kolmiot ovat keskenään
	yhdenmuotoiset.
	
	\numerointi{
		§ Kolmioissa on kaksi paria keskenään yhtä suuria vastinkulmia. (kk)
		§ Kolmioissa on kaksi paria keskenään verrannollisia sivuja, joiden välinen
		kulma on yhtä suuri molemmissa kolmioissa (sks)
		§ Kolmioissa on kolme paria keskenään verrannollisia sivuja. (sss)
	}
}

Vastinsivujen pituuksien suhde on jokin positiivinen vakio $k$. Koska
epäyhtälö $a < b$ säilyy kerrottaessa puolittain $k$:lla, niin vastinsivujen
suuruusjärjestyksen on oltava sama.

\begin{esimerkki}
Kolmiolla $ABC$ sivun $BC$ pituus on 4,2 cm, sivun $AC$ pituus on 3,5 cm ja sivun
$AB$ pituus on 5,6 cm. Onko $ABC$ yhdenmuotoinen sellaisen kolmion kanssa, jonka
sivujen pituudet ovat
\alakohdat{
	§ 6 cm, 5 cm ja 8 cm?
	§ 12,6 cm, 10,5 cm ja 16,8 cm?
	§ 2 cm, 4 cm ja 5 cm?
}
\begin{esimratk}
\alakohdat{
	Tarkastellaan kussakin kohdassa kolmioiden sivujen suhteita pienimmistä sivuista
	suurimpiin.
	§
		\begin{align*}
		\frac{3,5}{5} = 0,7 \\
		\frac{4,2}{6} = 0,7 \\
		\frac{5,6}{8} = 0,7
		\end{align*}
		Koska löydettiin vastinsivut, joiden suhteet ovat samat, kolmiot ovat
		yhdenmuotoiset.
	§
		\begin{align*}
		\frac{3,5}{10,5} = \frac{1}{3} \\
		\frac{4,2}{12,6} = \frac{1}{3} \\
		\frac{5,6}{16,8} = \frac{1}{3}
		\end{align*}
		Koska löydettiin vastinsivut, joiden suhteet ovat samat, kolmiot ovat
		yhdenmuotoiset.
	§
		\begin{align*}
		\frac{3,5}{2} = 1,75 \\
		\frac{4,2}{4} = 1,05 \\
		\frac{5,6}{5} = 1,12
		\end{align*}
		Suhteet eivät ole samat, joten kolmiot eivät ole yhdenmuotoiset.
}
\end{esimratk}
\begin{esimvast}
\alakohdat{
	§ On.
	§ On.
	§ Ei ole.
}
\end{esimvast}
\end{esimerkki}

\begin{esimerkki}
Kartan mittakaavaksi on merkitty $1:50\ 000$. Olli aikoo pyöräillä keskustaan tekemään
oppikirjaa. Kartalla pyöräreitti keskustaan on $24$ cm. Kuinka kauan Ollilla kestää
pyöräillä keskustaan, jos oletetaan, että hän pyöräilee keskimäärin $21$ km/h?
\begin{esimratk}
Merkitään $x$:llä pyöräreitin oikeaa pituutta senttimetreissä.
Kartoissa maastoa esittävät kuvat ovat yhdenmuotoisia todellisen maaston kanssa. Voimme
siis käyttää yhdenmuotoisuussuhdetta:
\begin{align*}
\frac{1}{50\ 000} = \frac{24}{x} \\
x = 50\ 000 \cdot 24 = 1\ 200\ 000
\end{align*}
Kilometreiksi muutettuna reitin pituus on siis 12 km. Nyt voidaan ratkaista matkaan kuluva
aika $t$:
\begin{align*}
\frac{12}{t} = 21 \\
t = \frac{12}{21} (\approx 0,57)
\end{align*}
Muutetaan tämä vielä minuuteiksi kertomalla luvulla 60:
\[
60 \cdot \frac{12}{21} = \frac{240}{7} \approx 34
\]
\end{esimratk}
\begin{esimvast}
34 minuuttia
\end{esimvast}
\end{esimerkki}

Huomaa, että säännössä (sks) yhtäsuuren vastinkulman on oltava vastinsivujen välissä. On
nimittäin mahdollista, että kahdella kolmiolla on kaksi keskenään verrannollista sivua
ja yhtä suuri kulma toisen sivun vieressä, mutta kolmiot eivät kuitenkaan ole yhdenmuotoiset.

\begin{esimerkki}
Vaikka kolmioilla $ABC$ ja $ABD$ on yhteinen kulma $\angle BAC = \angle BAD$, yhteinen sivu $AB$ ja yhtä pitkät sivut $BC = BD$, ei voida päätellä että kolmiot olisivat yhdenmuotoiset, sillä kulma ei ole vastinsivujen välissä eli (sks)-sääntöä ei voi käyttää.

\begin{kuva}
skaalaa(1.6)
A = geom.piste(0, 0, "$A$", -135)
B = geom.piste(2, 0, "$B$", -40)
w = geom.ympyra(B, 1.2, piirra = False)
s = geom.suora(A, (3, 2), piirra = False)
C = geom.leikkauspiste(s, w, "$C$", 40)
D = geom.leikkauspiste(s, w, "$D$", 120, valinta = 1)
geom.jana(A, B)
geom.jana(B, C)
geom.jana(C, D)
geom.jana(B, D)
geom.jana(D, A)
geom.kulma(B, A, C)
\end{kuva}

\end{esimerkki}
