\begin{tehtavasivu}

\begin{tehtava}
Todista kosinilauseen avulla Pythagoraan lauseen käänteislause: Jos kolmion sivuilla $a$, $b$ ja $c$ pätee
\[
a^2+b^2 = c^2,
\]
niin kolmio on suorakulmainen.

\begin{vastaus}
Vinkki: Mitä tiedät $\cos \gamma$ arvosta, kun sovellat kolmioon Pythagoraan lausetta ja kosinilausetta.
\end{vastaus}
\end{tehtava}

\begin{tehtava}
Kosinilauseen voi todistaa myös suoraan yhdenmuotoisista kolmioista lähtien. Olkoon $ABC$ tylppäkulmainen kolmio (Kulma $\alpha$ tylppä) ja pisteet $D$ ja $E$ janalla $BC$ siten, että $\angle BDA = \angle CAB = \angle AEC$.
\alakohdat{
§ Todista, että kolmio $ADE$ on tasakylkinen.
§ Todista, että kolmiot $ABC$, $DBA$ ja $EAC$ ovat yhdenmuotoisia.
§ Laske janojen $BD$ ja $EC$ pituudet kolmion $ABC$ sivujen pituuksien avulla.
§ Olkoon $M$ janan $ED$ keskipiste. Esitä janan $ED$ pituus kolmiosta kulman $\alpha$ ja kolmion $ABC$ sivujen pituuksien avulla hyödyntämällä syntyneitä suorakulmaisia kolmioita. Miksi ne ovat suorakulmaisia?
§ Nyt voit esittää sivun $BC$ pituuden osien $BD$, $DE$ ja $EC$ pituuksien summana, ja sievennys antaa tuloksen.
§ Mikä muuttuu, jos kolmio onkin teräväkulmainen?
}

\begin{vastaus}
\alakohdat{
§ Vinkki: Kulmat $ADE$ ja $DEA$ ovat yhtäsuuret. Miksi?
§ Vinkki: Hyödynnä yhtenmuotoisuussääntöä kk.
§ $BD = AB^2/BC$ ja $EC = AC^2/BC$. Vinkki: Hyödynnä yhdenmuotoisuussääntöä sss
§ $ED = -2 AB AC \cos{alpha}$. Vinkki: Kolmio $AED$ on tasakylkinen ja sen sivujen kylkien pituudet voidaan laskea edellisen kohdan tapaan. Myös kulman $DEA$ voi selvitää. Kolmiot $AEM$ ja $ADM$ ovat yhteneviä (sks), joten ED:n voi laskea kosinin määritelmän avulla.
§ Pisteet $E$ ja $D$ ovat eri järjestyksessä, joten merkit muuttuvat välivaiheissa hieman.
}
\end{vastaus}
\end{tehtava}

\end{tehtavasivu}
