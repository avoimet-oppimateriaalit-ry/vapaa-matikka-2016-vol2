\section*{Mitä on geometria?}

Geometria on tiede, joka tutkii tilaa ja muotoa. Sen keskeisiä käsitteitä ovat pisteet, suorat ja tasokuviot kuten ympyrät ja kolmiot. 

Antiikin Kreikassa joitakin geometrian tuloksia osattiin perustella muiden avulla. Eukleides kokosi aikansa geometrian teorian yhdeksi kokonaisuudeksi nimeltä Stoikheia eli Alkeet. Pyrkimyksenä oli luoda geometrialle teoreettinen pohja, jossa lähtemällä pienestä määrästä aksioomia ja postulaatteja voitaisiin osoittaa kaikki muu.

Uusimpien (vuoden 2003) lukion opetussuunnitelman perusteiden mukaan pitkän matematiikan ensimmäisen kurssin tavoitteena on, että opiskelija
\luettelo{
§ harjaantuu hahmottamaan ja kuvaamaan tilaa sekä muotoa koskevaa tietoa sekä kaksi- että kolmiulotteisissa tilanteissa
§ harjaantuu muotoilemaan, perustelemaan ja käyttämään geometrista tietoa käsitteleviä lauseita
§ ratkaisee geometrisia ongelmia käyttäen hyväksi kuvioiden ja kappaleiden ominaisuuksia, yhdenmuotoisuutta, Pythagoraan lausetta sekä suora- ja vinokulmaisen kolmion trigonometriaa.
}

Opetussuunnitelman perusteet määrittelevät kurssin keskeisiksi sisällöiksi
\luettelo{
§ kuvioiden ja kappaleiden yhdenmuotoisuuden
§ sini- ja kosinilauseen
§ ympyrän, sen osien ja siihen liittyvien suorien geometria
§ kuvioihin ja kappaleisiin liittyvien pituuksien, kulmien, pinta-alojen ja tilavuuksien laskeminen
}
