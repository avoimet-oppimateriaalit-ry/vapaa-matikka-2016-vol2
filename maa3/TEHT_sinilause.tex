\begin{tehtavasivu}

\paragraph*{Opi perusteet}

\begin{tehtava}
Kolmion sivujen pituudet 2, 3 ja 4. Määritä kolmion kulmien suuruudet. Mikä on kulmien summa?
\begin{vastaus}
Kulmien suuruudet ovat n. $29^\circ$, $47^\circ$ ja $104^\circ$. Kulmien summa on $180^\circ$.
\end{vastaus}
\end{tehtava}

\paragraph*{Hallitse kokonaisuus}

\begin{tehtava}
Kolmion yhden sivun pituus on 1, ja sitä vastaavan kulman suuruus $45^\circ$. Kuinka pitkiä ovat muut sivut, kun niiden pituuksien suhde on 3.

\begin{vastaus}
Muut sivut ovat pituudeltaan $\frac{1}{\sqrt{7}}$ ja $\frac{3}{\sqrt{7}}$.
\end{vastaus}
\end{tehtava}

\paragraph*{Sekalaisia tehtäviä}

\end{tehtavasivu}
