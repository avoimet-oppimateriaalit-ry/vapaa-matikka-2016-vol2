\begin{tehtavasivu}

\begin{tehtava}
Kultakalaparvi tarvitsee akvaarion, johon mahtuu vähintään 100 litraa vettä kutakin kalaa kohti. Jos haluaa pitää kymmentä kultakalaa pallonmuotoisessa kalamaljassa, kuinka suuri tämän kalamaljan säteen täytyy vähintään olla?
\begin{vastaus}
62cm
\end{vastaus}
\end{tehtava}

\begin{tehtava}
Oletetaan, että litra heliumia jaksaa kannatella yhden gramman verran painoa.
\alakohdat{
§ Kuinka monta grammaa jaksaa kannatella heliumpallo, jonka säde on 20cm?
§ Kaisa painaa 70kg ja myy heliumpalloja, joiden säde on 20cm. Kuinka monen pallon kimppu riittää kannattelemaan Kaisan painoa?
}
\begin{vastaus}
\alakohdat{
§ 34 grammaa
§ 2089 heliumpalloa
}
\end{vastaus}
\end{tehtava}

\begin{tehtava}
Tennispallon halkaisija on noin 7cm. Tennispalloja säilytetään pinossa lieriön muotoisessa, pahvisessa kotelossa. Kuinka paljon pahvia tarvitaan koteloon, johon mahtuu 5 tennispalloa? Saumoja ei tarvitse ottaa huomioon.
\begin{vastaus}
$850~\textnormal{cm}^2$
\end{vastaus}
\end{tehtava}

\begin{tehtava}
Kartioita ovat hatut, jäätelötötteröt, liikenteenohjauskartiot, megafonit, peitsenkärjet, mitä näitä nyt on
\begin{vastaus}
...
\end{vastaus}
\end{tehtava}

\begin{tehtava}
Osoita Pythagoraan lauseen yleistys kolmeen ulottuvuuteen: suorakulmaisen särmiön lävistäjän pituudelle $d$ pätee
$a^2+b^2+c^2=d^2$, missä $a$, $b$ ja $c$ ovat särmiön sivujen pituudet.
\end{tehtava}

\end{tehtavasivu}
