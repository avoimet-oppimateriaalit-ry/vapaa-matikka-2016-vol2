\begin{tehtavasivu}

\paragraph*{Opi perusteet}

\paragraph*{Hallitse kokonaisuus}

	\begin{tehtava}
Äiti avasi kukkaronsa nyörejä ja ostaa paukautti suuren taulutelevision, jonka mitat ovat $300$\,cm \times $210$\,cm. Mahtuuko televisio edes sisään ulko-ovesta, jonka mitat on $200$\,cm \times $80$\,cm?
	\begin{vastaus}
	 Kyllä mahtuu. (Oven lävistäjä on suurempi kuin telkkarin korkeus.)
	\end{vastaus}
	\end{tehtava}
	
	\begin{tehtava}
Suorakulmion muotoisen vanerilevyn mitat ovat $230\,\text{cm}\times 250\,\text{cm}$. Mahtuuko se sisään oviaukosta, joka on $90$\,cm leveä ja $205$\,cm korkea?
        \begin{vastaus}
        Ei mahdu. Suurin mahdollinen levy, joka mahtuu ovesta sisään on Pythagoraan lauseen perusteella leveydeltään $\sqrt{90^2+205^2}\approx 224$\,cm.
        \end{vastaus}
\end{tehtava}

\begin{tehtava}
$\star$ Ajatellaan suorakulmaista hiekkakenttää, jonka pinta-ala on $1$ aari ($100\,\mathrm{m}^2$). Lyhyemmän ja pidemmän sivujen pituuksien suhde on $4:3$. Laske Pythagoraan lauseen avulla matka hiekkakentän kulmasta kauimmaisena olevaan kulmaan.
\begin{vastaus}
$\frac{4}{3}x^2=100$, joten $x = \sqrt{\frac{300}{4}}$. 
Hypotenuusa: $\sqrt{x^2 + (\frac{4}{3}x)^2}=\sqrt{\frac{300}{4}+\frac{16}{9}\cdot \frac{300}{4}}
=\frac{5}{3}\sqrt{\frac{300}{4}}\approx 14,4$.
\end{vastaus}
\end{tehtava}

\begin{tehtava}
Todista Pythagoraan lauseen käänteislause: Jos kolmion sivujen pituuksille $a,b$ ja $c$ pätee
\[
a^2+b^2 = c^2,
\]
niin kolmio on suorakulmainen.

\begin{vastaus}
Vinkki: Tutki suorakulmaista kolmiota, jonka kateettien pituudet ovat $a$ ja $b$ ja hyödynnä yhtenevyyttä.
\end{vastaus}
\end{tehtava}

\begin{tehtava}
Kolmion ala voidaan esittää myös suoraan kolmion sivujen pituuksien funktiona. Kolmion $ABC$ sivua $BC$ vastaavan korkeusjanan $AD$ pituus voidaan määrittää käyttämällä Pythagoraan lausetta kolmioihin $ABD$ ja $ACD$.
\alakohdat{
§ Lausu sivun $BC$ pituus sivujen $AC$ ja $AB$ ja korkeusjanan $AD$ pituuksien avulla.
§ Ratkaise korkeusjanan pituuden neliö neliöimällä yhtälö ja sieventämällä kahdesti.
§ Nyt voit muodostaa polynomilausekkeen kolmion pinta-alan neliölle.
§ Merkitse vielä $BC = a$, $AC = b$, $AB = c$, sekä $p = \frac{a+b+c}{2}$ ja todista kuuluisa Heronin kaava kolmion pinta-alalle
\[
A_{ABC} = \sqrt{p(p-a)(p-b)(p-c)}
\]
}

\begin{vastaus}
\alakohdat{
$BC = \pm\sqrt{AC^2-AD^2}\pm\sqrt{AB^2-AD^2}$. $\pm$-merkit valitaan riippuen kolmion tyypistä.
$AD^2 = \frac{4AC^2AB^2-(BC^2-AC^2-AB^2)^2}{BC^2}$
$A_{ABC}^2 = \frac{4AC^2AB^2-(BC^2-AC^2-AB^2)^2}{16}$
Vinkki: Käytä toistuvasti identiteettiä $x^2-y^2 = (x+y)(x-y)$ tai kerro raa'asti auki.
}
\end{vastaus}
\end{tehtava}

\paragraph*{Sekalaisia tehtäviä}

\begin{tehtava}
	Ratkaise puuttuvat sivut
	\alakohdat{
		§ Kateetit ovat $3$ ja $4$
		§ Kateetit ovat $1$ ja $2$
		§ Toinen kateetti, kun hypotenuusa on $\sqrt{2}$ ja toinen kateetti on $1$
		§ Hypotenuusa on $8$, kun molemmat kateetit ovat yhtä pitkiä
	}
	\begin{vastaus}
		\alakohdat{
			$5$
			$\sqrt{5}$
			$1$
			$2\sqrt{2}$
		}
	\end{vastaus}
\end{tehtava}

\begin{tehtava}
Todista, että jos kolmion sivujen pituudet ovat $2ab$, $a^2-b^2$ ja $a^2+b^2$ ($a>b$), niin kolmio on suorakulmainen.
\end{tehtava}

\end{tehtavasivu}
