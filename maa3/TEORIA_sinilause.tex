\section{Sinilause}

\begin{luoKuva}{sinilause}
A = geom.piste(0, 0, piirra = False)
B = geom.piste(5, 2, piirra = False)
C = geom.piste(2, 4, piirra = False)
geom.jana(A, B, "$c$", kohta = 0.35)
geom.jana(B, C, "$a$")
geom.jana(C, A, "$b$")
geom.kulma(B, A, C, r"$\alpha$")
geom.kulma(C, B, A, r"$\beta$")
geom.kulma(A, C, B, r"$\gamma$")

w = geom.ymparipiirrettyYmpyra(A, B, C)
O = geom.ympyranKeskipiste(w)
R = geom.ympyranKehapiste(w, -20, piirra = False)
geom.jana(O, R, "$R$", kohta = 0.6)
\end{luoKuva}

\laatikko{
Kolmion sivut ovat $a$, $b$ ja $c$, ja niitä vastaavat kulmat ovat $\alpha$, $\beta$ ja $\gamma$. \termi{sinilause}{Sinilauseen} mukaan
$$\frac{a}{\sin \alpha} = \frac{b}{\sin \beta} = \frac{c}{\sin \gamma} = 2R,$$
missä $R$ on kolmion ympäri piirretyn ympyrän säde.

\begin{center}
\naytaKuva{sinilause}
\end{center}
}

\begin{center}
 \begin{kuva}
	rajaa(minX = -1, maxX = 6)
	A = (0, 0)
	B = (5, 0)
	C = (4, 3)
	P = (4, 0)
	geom.jana(A, B, "$c$")
	geom.jana(B, C, "$a$")
	geom.jana(C, A, "$b$")
	geom.jana(P, C, "$h$")
	geom.kulma(B,A,C,r"$\alpha$")
	geom.kulma(C,B,A,r"$\beta$")
	vari("black!10!white")
	geom.jana(P, C)
	geom.suorakulma(C, P, A)
	geom.jana(B, P)
\end{kuva}
\end{center}


\textbf{Sinilauseen todistus.} Tarkastellaan pisimmälle sivulle $c$ piirrettyä korkeusjanaa.
Sen pituus $h$ on suorakulmaisen kolmion kateettina, joten sen voi ilmoittaa muodossa
\[
h = a \sin \beta
\]
tai käyttämällä toista suorakulmaista kolmiota, muodossa
\[
h = b \sin \alpha .
\]

Tällöin on voimassa
\[
a \sin \beta = b \sin \alpha
\]
eli
\[
\frac{a}{\sin \alpha} = \frac{b}{\sin \beta}.
\]

Vastaavalla tavalla piirtämällä sivulle $a$ korkeusjanan saisi pääteltyä, että
\[
\frac{b}{\sin \beta} = \frac{c}{\sin \gamma}.
\]

Viimeisen yhtäsuuruuden voimassaolo perustellaan myös kehäkulmalauseen yhteydessä.
%onko esitysjärjestys tämä ja perustellaanko tuo siellä?
