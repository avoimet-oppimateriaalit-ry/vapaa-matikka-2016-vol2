\begin{tehtavasivu}

\paragraph*{Opi perusteet}

\paragraph*{Hallitse kokonaisuus}

\begin{tehtava}
Ympyränmuotoisen lehden reunalla oleva toukka näkee $8^\circ$ asteen kulmassa vastakkaisella 
laidalla olevan $4$ mm pitkän kuihtuneen osan. Mikä on lehden läpimitta?
\begin{vastaus}
$1,4$ cm
\end{vastaus}  
\end{tehtava}

\begin{tehtava}
Brutopian armeija on rynnäköimässä Naalin niemimaalla olevaan tukikohtaan. Radioviestejä kuuntelemalla tukiohdassa tiedetään, että brutopialaisesta panssarivaunusta tukikohdan $30$ metriä leveä seinämä näkyy $10^\circ$ kulmassa.
Missä brutopialainen panssarivaunu sijaitsee?
\begin{vastaus}
Ympyrän kehällä, jonka keskipiste on noin $85$ metrin päässä tukikohdan seinämän keskipisteestä, jonka säde on noin $86$ metriä.
%kuva
\end{vastaus}
\end{tehtava}

\begin{tehtava}
Nelikulmio on \termi{jännenelikulmio}{jännenelikulmio}, jos sen kaikki kärjet ovat samalla ympyrällä. Todista jännenelikulmiolause: Nelikulmio on jännenelikulmio jos ja vain jos sen vastakkaisten kulmien summa on $180^{\circ}$.
\begin{vastaus}
Vinkki: $\Rightarrow$ Käytä kehäkulmalausetta. $\Leftarrow$ Piirrä ympyrä, tutki missä sivujen jatkeet leikkaavat ympyrän, ja hyödynnä lauseen toista puolta.
\end{vastaus}
\end{tehtava}

\begin{tehtava}
Pisteestä $P$ piirretyt suorat $l_1$ ja $l_2$ leikkaavat ympyrän $\Gamma$ pisteissä $A_1$, $B_1$, sekä $A_2$ ja $B_2$, vastaavasti. Jos suora $l_{i}$ sivuaa ympyrää, $A_{i} = B_{i}$. Todista, että
\[
PA_{1}\cdot PB_{1} = PA_{2}\cdot PB_{2}
\]
Huom.: Piste $P$ voi olla ympyrän sisäpuolella, ulkopuolella, tai ympyrällä.
\begin{vastaus}
Vinkki: Mitä voit sanoa kolmioista $PA_1A_2$ ja $PB_1B_2$?
\end{vastaus}
\end{tehtava}

\end{tehtavasivu}
