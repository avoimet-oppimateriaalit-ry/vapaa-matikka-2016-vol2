\subsection{Vektorin esitystapoja}

Vektoreita voidaan esittää erillaisilla tavoilla. Tietyn esitystavan käyttö joissakin tilanteissa helpottaa vektoreilla laskemista. Tässä luvussa esitellään lyhyesti vektorien koordinaattiesitys. Vektoreilla laskemiseen ja koordinaattiesityksen käyttöön palataan vielä tarkemmin tulevissa luvuissa.

Koska vektorin määrittää sen suunta ja pituus, ei sen alkupisteellä ole väliä. Vektori voidaan siis aina halutessa valita alkamaan origosta. Origosta alkavaa vektoria kutsutaan \termi{paikkavektori}{paikkavektoriksi}. Mikäli vektori $\bar{a}$ on annettu muodossa $\bar{a} = \overrightarrow{AB}$, missä pisteen
$A$ koordinaatit $xy$-tasossa ovat $(x_1,y_1)$ ja pisteen $B$ koordinaatit $(x_2,y_2)$, voidaan $\bar{a}$ esitettää paikkavektorina muodossa $\bar{a} = \overrightarrow{OP}$, missä $O=(0,0)$ on origo ja pisteen $P$ koordinaatit ovat $(x_2-x_1,y_2-y_1)$. Kurssin MAA4 tietoja käyttäen voi varmistaa, että vektorit $\overrightarrow{AB}$ ja $\overrightarrow{OP}$ ovat todellakin samansuuntaisia ja samanpituisia ja siten samoja vektoreita.

(Kuva vektoreista $\overrightarrow{AB}$ ja $\overrightarrow{OP}$, josta näkyy, että ne ovat samansuuntaisia, samanpituisia ja $\overrightarrow{OP}$ alkaa origosta.)

Vektorin $\bar{a}$ paikkavektoriesitys $\bar{a} = \overrightarrow{OP}$ on yksikäsitteinen ja sen määrittämiseen riittää tieto pisteen $P$ koordinaateista. Tässä mielessä vektori $\bar{a} = \overrightarrow{OP}$ voidaan samaistaa pisteen $P$ kanssa. Tähän liityen käytettään vektoreille usein merkintää 
\[
\bar{a} = \left( \begin{array}{c}
x \\
y 
\end{array}
\right), \, \, \text{missä $x$ ja $y$ ovat pisteen $P$ koordinaatit.}
\]
